\documentclass{sprawozdanie-agh}


\usepackage[utf8]{inputenc}
\usepackage{listings}
\usepackage{pdfpages}
\usepackage{float}
\usepackage{anyfontsize}
 
\makeatletter 

\begin{document}   

	\przedmiot{Inżynieria oprogramowania}
	\tytul{Sprawozdanie z projektu}
	\podtytul{„Gdzie jest moje dziecko?”}
	\kierunek{Informatyka, III rok, 2018/2019}
	\autor{Agnieszka Zadworny, Piotr Morawiecki, Tomasz Pęcak, Maciej Bielech}
	\data{Kraków, 7 listopada 2018}

	\stronatytulowa{}

	\section{Ogólny opis systemu}

		\subsection{Cel systemu}

			Celem naszej aplikacji, jest udostępnienie rodzicom możliwości kontroli lokalizacji swoich dzieci. Mogą oni tworzyć reguły, składające się z informacji o obszarze, w którym dziecko powinno przebywać w danym czasie. Jeśli dziecko złamie regułę, rodzic zostaje powiadomiony poprzez wiadomość email. W przypadku wyłączenia aplikacji dziecka, rodzic zostaje powiadomiony oraz może wyświetlić ostatnią zarejestrowaną lokalizację dziecka.

		\subsection{Udziałowcy i ich cele}

			Użytkownikami systemu są rodzice, którzy chcą zadbać o bezpieczeństwo swoich dzieci. Udziałowcami systemu są:

			\begin{itemize}
				\item System Google OAuth, który odpowiada za autentyfikację użytkownika w aplikacji,
				\item System Google Maps, który odpowiada za zarządzanie i wyświetlanie lokalizacji dziecka.
			\end{itemize}
			Biernymi użytkownikami systemu są dzieci, których aplikacja mobilna dostarcza jedynie informację o ich bieżącym położeniu.

			\begin{table}[h]
				\begin{center}
					\begin{tabular}{|c|p{7cm}|c|}
						\cline{1-3}
						\textbf{Udziałowiec} & \textbf{Cel} & \textbf{Priorytet} \\
						\cline{1-3}
						Rodzic & tworzenie reguł & wysoki \\
						\cline{1-3}
						Rodzic & sprawdzanie bieżącej lokalizacji dziecka & wysoki \\
						\cline{1-3}
						Dziecko & udostępnienie swojej lokalizacji & wysoki \\
						\cline{1-3}
						Google OAuth & udostępnienie API logowania & niski \\
						\cline{1-3}
						Google Maps & wyświetlanie lokalizacji dziecka na mapie & wysoki \\
						\cline{1-3}
						Google Maps & udostępnienie API zarządzania lokalizacją & wysoki \\
						\cline{1-3}
						SendGrid & wysłanie emaila do rodzica z powiadomieniem o złamaniu reguły przez dziecko  & wysoki \\
						\cline{1-3}
					\end{tabular}
				\end{center}
				\caption{Udziałowcy i ich cele}
			\end{table}

		\subsection{Granice systemu}

			Wyróżniamy następujących aktorów:
			\begin{itemize}
				\item Rodzic,
				\item Dziecko,
				\item System autentyfikacji Google OAuth,
				\item System Google Maps,
				\item System do wysyłansia emaili SendGrid,
				\item Czas.
			\end{itemize}

		\subsection{Lista możliwości}

			Lista możliwości została przedstawiona w postaci diagramów aktywności. Diagramy te reprezentują typowe sekwencje działań wykonywanych w systemie. W kolejnych sekcjach zostaną one zaprezentowane w postaci przypadków użycia i scenariuszy.

			\begin{figure}[H]
				\centering
				\begin{tabular}{c}
					\includegraphics[width=.95\textwidth]{cropped_Registration_Activity_Diagram} 
				\end{tabular}
			\caption{Diagram aktywności dla rejestracji}
			\end{figure}

			\begin{figure}[H]
				\centering
				\begin{tabular}{c}
					\includegraphics[width=.95\textwidth]{Login} 
				\end{tabular}
			\caption{Diagram aktywności dla logowania}
			\end{figure}

			\begin{figure}[H]
				\centering
				\begin{tabular}{c}
					\includegraphics[width=.95\textwidth]{crudCreate} 
				\end{tabular}
			\caption{Diagram aktywności dla tworzenia nowej reguły}
			\end{figure}

			\begin{figure}[H]
				\centering
				\begin{tabular}{c}
					\includegraphics[width=.95\textwidth]{crudUpdate} 
				\end{tabular}
			\caption{Diagram aktywności dla modyfikacji reguły}
			\end{figure}

			\begin{figure}[H]
				\centering
				\begin{tabular}{c}
					\includegraphics[width=.95\textwidth]{crudDelete} 
				\end{tabular}
			\caption{Diagram aktywności dla usuwania reguły}
			\end{figure}

			W diagramach wyróżniliśmy operacje CRUD (tworzenie, odczyt, modyfikacja i usuwanie) dla zarządzania regułami. Nie umieszczaliśmy pozostałych, gdyż realizowane są one w analogiczny sposób.

	\section{Analiza dziedziny}

		Klasy zidentyfikowane w aplikacji:
		\begin{itemize}
			\item KontoRodzica - klasa odpowiadająca za przechowywanie informacji o rodzicu takich jak imię 
		\end{itemize}

	\section{Specyfikacja wymagań}

	\section{Architektura systemu}

	\section{Projek oprogramowania}

	\section{Projekt interfejsu użytkownika}

	\section{Projekt bazy danych}

\end{document}